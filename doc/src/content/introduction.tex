%----------------------------------------------------------------------------
\chapter{\bevezetes}
%----------------------------------------------------------------------------

Szoftverrendszerek, különösen biztonságkritikus rendszerek esetében az egyik legfontosabb feladat az elkészült rendszer validációja. Ehhez számtalan megközelítés, módszer használatos az iparban. Ezek közül önálló laboratórium során a modellalapú tesztelést vizsgáltam, amelyet főként biztonságkritikus rendszereknél használnak.

Tekintettel a probléma fontosságára és a projektek számára nem elhanyagolható határidőkre, fontos, hogy a hatékony eszközt válasszuk a probléma megoldására.

Ezért az önálló laboratóriumi foglalkozásom során két eszközt, egy iparban elterjedt, nyílt forráskódú és egy új, kevésbé ismert eszközt vizsgáltam meg egy közös esettanulmány segítségével. Ezen esettanulmány által végeztem összehasonlítást a két eszközről.
