\chapter{Esettanulmány}
\section{Felhasznált eszközök}
A SysML v1-es részhez a Sparx Systems által fejlesztett Enterprise Architect eszközt használtam.
A SysML v2-es részhez a SysML v2 Submission Team (SST) által kiadott pilot-implementation-t használtam, Jupyter notebook formájában.

\section{SysML v2 telepítése}
Mivel a SysML v2 még érte el az első hivatalos kiadását, ezért nincs is hozzá hivatalos telepítő.
Ehelyett az SST havi rendszerességgel szolgáltat referencia implementációt GitHub-on keresztül\footnote{\url{https://github.com/Systems-Modeling/SysML-v2-Release}}.

Önálló laborom végrehajtása során a nyelv jupyter notebookhoz integrált kerneljével ismerkedtem.
A jupyter notebook egy interaktív felületet biztosító webszolgáltatás, ami képes Python kód futtatására.

Ehhez a fejlesztők készítettek ugyan egy telepítő szkriptet, ami a Miniconda keretrendszer felhasználásával - egy ingyenes csomagkezelő és környezet menedzsment rendszer\footnote{\url{https://docs.conda.io/en/latest/miniconda.html}} - telepíti a felhasználó számára a SysML v2 jupyter kernelt.

A mérés során fény derült azonban egy olyan hibára, hogy ha a számítógép tartalmaz több python könyvtárat, a miniconda által feltelepítetten kívül, akkor a jupyter szerver indítása után nem lehetséges kiválasztani a SysML kernelt ezáltal használhatatlanná téve azt.

A probléma egyik megoldása, melyet a labor végrehajtása során alkalmaztam, hogy a jupyter notebookot Visual Studio Code-ban futtattam és így érdekes módon ki tudtam választani futtatási kernelnek a SysML implementációját.

\section{Funkcionális dekompozíció}

\section{Viselkedés modellezés}

\section{Allokáció}